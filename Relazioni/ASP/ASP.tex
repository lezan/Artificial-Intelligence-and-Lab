\documentclass[12pt]{report}
\usepackage[a4paper]{geometry}
\usepackage[myheadings]{fullpage}
\usepackage{fancyhdr}
\usepackage{lastpage}
\usepackage{graphicx, wrapfig, subcaption, setspace, booktabs}
\usepackage[T1]{fontenc}
\usepackage[font=small, labelfont=bf]{caption}
\usepackage{fourier}
\usepackage[protrusion=true, expansion=true]{microtype}
\usepackage{sectsty}
\usepackage{url, lipsum}
\usepackage{listings}
\usepackage[utf8]{inputenc}
\usepackage[italian]{babel}

\lstset{
  basicstyle=\ttfamily,
  columns=fullflexible,
  frame=single,
  breaklines=true,
}

\newcommand{\HRule}[1]{\rule{\linewidth}{#1}}
\onehalfspacing
\setcounter{tocdepth}{5}
\setcounter{secnumdepth}{5}

%-------------------------------------------------------------------------------
% HEADER & FOOTER
%-------------------------------------------------------------------------------
\pagestyle{fancy}
\fancyhead{}
\setlength\headheight{15pt}
\fancyhead[L]{Intelligenza Artificiale e Laboratorio}
\fancyhead[R]{Universit\`a degli Studi di Torino}
%\fancyfoot[R]{Page \thepage\ of \pageref{LastPage}}
%-------------------------------------------------------------------------------
% TITLE PAGE
%-------------------------------------------------------------------------------

\begin{document}

\title{ \normalsize \textsc{}
        \\ [2.0cm]
        \HRule{0.5pt} \\
        \LARGE \textbf{\uppercase{Intelligenza Artificiale e Laboratorio: \\ Answer Set Programming \\e\\ Clingo}}
        \HRule{2pt} \\ [0.5cm]
        \normalsize \today \vspace*{5\baselineskip}}

\date{}

\author{
        Andrea Forgione: 809240 \\ 
        Arthur Capozzi:  802586 \\
        Leonardo Zanchi: 800935}

\maketitle
\tableofcontents
\newpage

%-------------------------------------------------------------------------------
% Section title formatting
\sectionfont{\scshape}
%-------------------------------------------------------------------------------

%-------------------------------------------------------------------------------
% BODY
%-------------------------------------------------------------------------------

\chapter*{Introduzione}

In questa relazione saranno presentati gli esercizi riguardandi la parte di \emph{Answer Set Programming} e Clingo.
I domini scelti sono due: le cinque case e il trasporto aereo di merci.
Il lavoro sarà diviso quindi in due capitoli, uno per ciascun problema considerato.
La versione di Clingo utilizzata è \emph{Clingo 5.2.1 WIN64}.

\chapter{Cinque case}
Il dominio delle cinque case ha come obiettivo quello di trovare chi tra le persone descritte ha come animale preferito la zebra.
La soluzione si trova all'interno del file \emph{house\_problem.lp}.
La struttura del programma segue il design \emph{Generate-Define-Test}, e quindi l'implementazione si articola in tre sezioni differenti, ognuna con le sue caratteristiche: 
\begin{itemize}
\item \emph{Generate};
\item \emph{Define};
\item \emph{Test};
\end{itemize}

\section{Generate}
Questa sezione definisce una collezione di \emph{answer sets} che possono essere delle potenziali soluzioni al problema. Aggregati e \emph{choice rules} sono tipicamente contenute in questa sezione.
Con le regole riportate nel listato, si definisce che le caratteristiche come colore, nazionalità, animale, bevanda e professione siano uniche, cioè, per esempio, può esistere al più una casa con un stesso colore. 
Queste condizioni sono espresse attraverso delle \emph{cardinality constraint}, che, sempre in riferimento al colore, stabiliscono di assegnare mediante la funzione \emph{colore} ad ogni fatto \emph{case} uno (ed un solo) fatto \emph{colori}. Con la seconda regola, invece, si afferma il contario, cioè che ad ogni fatto \emph{colori} corrisponda un solo fatto \emph{case}.
Come già detto, il medesimo ragionamento vale anche per le altre caratteristiche.

\begin{lstlisting}[language=lisp]
% Colore
1 { colore(Casa, Colore) : colori(Colore) } 1 :- case(Casa).
1 { colore(Casa, Colore) : case(Casa) } 1 :- colori(Colore).
\end{lstlisting}

\section{Define}
Questa parte esprime concetti aggiuntivi e connette la parte \emph{generate} a quella \emph{test}.
Sono riportati tutti i fatti che l'enunciato descrive:

\begin{itemize}
\item le nazionalità dei personaggi (definita \emph{nazionalitas} per conflitto e mancanza di fantasia sul nome);
\item le case, definite con un intervallo, assegnando a ciascuna un numero;
\item gli animali, le bevande ed il colore;
\end{itemize}

È stata inserita anche una regola per definire l'adiacenza di una casa all'altra. Per esempio \emph{casa(1)} e \emph{casa(2)} sono adiacenti perché $|1 - 2 | == 1$, mentre \emph{casa(5)} e {casa(3)} perché $|5 - 3| != 1$.
Per convenzione si è stabilito che la casa all'estrema sinistra sia identificata dal numero $1$, mentre quella all'estrema destra da $5$. Inoltre, la casa \emph{X} si trova a sinistra della casa \emph{Y} se $X < Y$, viceversa se la casa \emph{X} si trova a destra della casa \emph{Y}, allora $X > Y$.\\
È presente anche un'altra regola, \emph{chi\_ha\_la\_zebra(Nazionalita)}, che ha la funzione di restituire la risposta cercata. La regola ha due goal:

\begin{itemize}
\item \emph{nazionalita(Casa, Nazionalita)}, che consente di connettere la nazionalità alla casa;
\item \emph{animale(Casa, zebra)}, che permette di fare il passo finale, cioè legare la casa all'animale di nostro interesse;
\end{itemize}

\begin{lstlisting}[language=lisp]
% Nazionalita'
nazionalitas(inglese;spagnolo;giapponese;italiano;norvegese).

% Casa
case(1..5).

% Funzione di adiacenza
next_to(X, Y) :- case(X), case(Y), |X - Y| == 1.

chi_ha_la_zebra(Nazionalita) :- nazionalita(Casa, Nazionalita), animale(Casa, zebra).
\end{lstlisting}

\section{Test}
Questo modulo consiste in regole che eliminano gli \emph{answer sets} della parte \emph{generate} che non possono essere delle soluzioni.
L'implementazione contiene tutti i vincoli che sono espressi dall'enunciato.
Per esempio, il fatto $1$ è espresso dal vincolo riportato nel listato, dove si escludono dall'\emph{answer set} le soluzioni che hanno il personaggio inglese in una casa che non è rossa.
Un discorso analogo può essere fatto per la regola $14$, dove invece si escludono dall'\emph{answer set} le soluzioni che prevedono che il cavallo non sia adiacente alla casa del diplomatico.

\begin{lstlisting}[language=lisp]
% 1. L'inglese vive nella casa rossa.
:- nazionalita(X, inglese), colore(X, Y), Y != rossa.
% 14. Il cavallo e' nella casa adiacente a quella del diplomatico
:- animale(X, cavallo), professione(Y, diplomatico), not next_to(X, Y).
\end{lstlisting}

\section{Output}
Questa parte finale non è propriamente definita dal \emph{design structure Generate-Define-Test}. 
Ha lo scopo di restituire la soluzione trovata.
La regola \emph{casa/6} realizza una sorta di tabella, che, attraverso l'istruzione \emph{\#show casa/6}, permette di essere stampata ed avere una visione di insieme su quello che si è ottenuto dall'esecuzione.
L'istruzione \emph{\#show chi\_ha\_la\_zebra/1} resituisce la nazionalità del possessore di questo animale utilizzando la regola definita nella sezione \emph{define}.
La soluzione restituita da Clingo dopo l'esecuzione di \emph{clingo 0 house\_problem.lp} è riportata nel listato.

\begin{lstlisting}[language=lisp]
chi_ha_la_zebra(giapponese)
casa(1,gialla,norvegese,volpe,altro,diplomatico)
casa(2,blu,italiano,cavallo,te,dottore)
casa(3,rossa,inglese,lumache,latte,scultore)
casa(4,bianca,spagnolo,cane,succo_di_frutta,violinista)
casa(5,verde,giapponese,zebra,caffe,pittore)
SATISFIABLE
\end{lstlisting}

\chapter{Trasporto aereo di merci}
Questo capitolo prende in esame il problema del trasporto aereo di merci, tratto dal \emph{Russell e Norvig}, capitolo 10.1.
La soluzione si trova all'interno del file \emph{plane\_problem.lp}.\\
A differenza della precedente soluzione, questa segue una differente struttura, divindendosi in tre parti: la prima è introdotta dalla direttiva \emph{\#program base}, la seconda da \emph{\#program step(t)} e la terza da \emph{\#program check(t)}. \emph{Clingo}, infatti, consente la possibilità di dividere le regole di input in subprograms attraverso la direttiva \emph{\#program}.
L'utilizzo di `\emph{t}` è di fondamentale importanza, infatti essa mantiene lo stato del programma, consentendo una esecuzione incrementale del problema. 
Per poterne fare uso è necessario dichiarare la volontà di utilizzarlo e questo è possibile inserendo la libreria \emph{incmode} attraverso l'istruzione \emph{\#include <incmode>.}.

\section{Sezione base}
Questa sezione è una parte dedicata del \emph{subprogram} con una lista dei parametri vuota.
Contiene essenzialmente una lista di fatti che definiscono il dominio del problema. Sono riportati infatti l'elenco degli oggetti presenti (cargo, plane e airport), lo stato iniziale e lo stato goal.
È presente, inoltre, il caso base della regola di inerzia che permette di risolvere il problema. Si è deciso di utilizzare il predicato \emph{holds(F, 0)} che racchiude il fluente \emph{F}, che può essere sia `\emph{in}` che `\emph{at}`, così da avere una maggiore generalità, evitando di specificare le singole regole di inerzia per ciascuno.

\section{Sezione step}
Questa sezione è la più corposa e contiene al suo interno le regole che consentono di risolvere il problema descritto dal dominio.
Per comodità si può dividere questa sezione in quattro parti.

\paragraph*{Choice rules}
La prima parte consiste in una singola \emph{choice rule} che consente di speficare che è possibile una sola azione ogni step.

\begin{lstlisting}[language=lisp]
1 { load(C, P, A, t): cargo(C), plane(P), airport(A); unload(C, P, A, t): cargo(C), plane(P), airport(A); fly(P, FROM, TO, t): plane(P), airport(FROM), airport(TO), FROM != TO} 1.
\end{lstlisting}

\paragraph*{Inerzia}
La seconda parte contiene le regole di inerzia del predicato \emph{holds} in cui si specifica che il fluente non cambia il suo stato se non è esplicamente rimosso.
La prima regola, per esempio, specifica che per avere il fluente \emph{F} nello stato \emph{t} del predicato \emph{holds}, si aveva il fluente nello stato \emph{t - 1} e che c'è assenza di informazione sulla negazione del predicato nello stato \emph{t}.

\begin{lstlisting}[language=lisp]
holds(F, t) :- holds(F, t - 1), not -holds(F, t).
-holds(F, t) :- -holds(F, t - 1), not holds(F, t).
\end{lstlisting}

\paragraph*{Effetti}
La terza parte contiene gli effetti delle azioni di \emph{load}, \emph{unload} e \emph{fly}.
Nel listato sono riportati gli effetti che l'azione \emph{fly} comporta: volando dall'aereoporto \emph{FROM} all'aereoporto \emph{TO} nello stato \emph{t}, l'aereoplano \emph{P} si troverà nell'aereoporto \emph{TO} e non nell'aereoporto \emph{FROM} nello stato \emph{t}.

\begin{lstlisting}[language=lisp]
holds(at(P, TO), t) :- fly(P, FROM, TO, t).
-holds(at(P, FROM), t) :- fly(P, FROM, TO, t).
\end{lstlisting}

\paragraph*{Vincoli}
L'ultima parte contiene le precondioni sotto le quali le azioni sono possibili.
Per esempio, nel listato riportato si esclude l'operazione di \emph{load} nello stato \emph{t} se avevo compiuto la stessa operazione sullo stesso cargo nello stato precedente \emph{t - 1}, qualsiasi sia l'aereoplano.

\begin{lstlisting}[language=lisp]
:- load(C, P, A, t), holds(in(C, _), t - 1).
\end{lstlisting}

\section{Check}
Questa terza sezione verifica in ogni stato se si è raggiunto lo stato goal.

\begin{lstlisting}[language=lisp]
:- query(t), goal(F), not holds(F, t).
\end{lstlisting}

\section{Output}
La soluzione proposta da \emph{Clingo} a questo problema è riportata nel listato. Quella inserita è la migliore che riesce a trovare sotto i vincoli espressi precedentemente. In totale \emph{Clingo} invidivuda $22$ differenti modelli di \emph{answer set}.

\begin{lstlisting}[language=lisp]
load(c1,p1,sfo,1)
fly(p1,sfo,jfk,2)
load(c2,p2,jfk,3)
fly(p2,jfk,sfo,4)
unload(c2,p2,sfo,5)
unload(c1,p1,jfk,6)
SATISFIABLE
\end{lstlisting}

\end{document}